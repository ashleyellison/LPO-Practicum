\documentclass[12pt]{article}
\usepackage{bibentry}
\usepackage{fourier}
\usepackage[margin=1in]{geometry}

\usepackage[colorlinks=true,
                      pdfstartview=FitV,
                      urlcolor=blue,
]{hyperref}
\usepackage{natbib}



\begin{document}
\bibliographystyle{apalike}
\nobibliography{nature}

\thispagestyle{empty}%


\setlength{\parskip}{1ex plus 0.5ex minus 0.2ex}

\setcounter{secnumdepth}{-2}



\begin{flushleft}
Vanderbilt University\\
Leadership, Policy and Organizations\\
Class Number 9952\\ 
Spring 2019\\
\end{flushleft}

\begin{center}
\Large{\textbf{Methods Practicum}}\\
\end{center}

\begin{flushleft}
William R. Doyle\\
Office: 207D Payne\\
Office Hours: Mondays and Wednesdays 2-5 pm or by Appointment \\
w.doyle@vanderbilt.edu\\
phone (615) 322-2904\\

\vskip 12 pt

\end{flushleft}

\section{Course Overview}%

The overview includes an introduction to the course, guidelines on grading, and required texts.

\subsection{Introduction}%
\begin{flushleft}

  This course is the second of a three semester series of courses
  designed to introduce you to the \textit{practice} of research,
  particularly the applied side of quantitative research. The goal of
  this course to help you to prepare a paper that can be presented at
  a major research conference and, hopefully, submitted to a journal
  for publication.


  During this semester you will use multiple regression in order to
  analyze the research question you identified last semester.
  
\end{flushleft}

\subsection{Grading}%
\begin{flushleft}
Evaluation for the course will be based on the following factors:

\textit{Assignments: 50\%}

There will be weekly assignments, which will be graded. Late assignments will not be accepted. These assignments
will account for half of your grade. Collaboration on assignments is
fine, however, many of the assignments will ask you to work with
variables and datasets of your own choosing.

\textit{Replication File: 50\%}


\end{flushleft}

\subsection{Texts}%

\begin{flushleft}

Baum, C (2006) \textit{An Introduction to Modern Econometrics Using
STATA}. College Station: STATA Press

Long, J.S. (2009) \textit{The Workflow of Data Analysis Using Stata}. College Station: STATA Press

For this semester, both of these books are \textit{optional}: I will
recommend a few chapters from both as the semester progresses, but the
class notes are the only required reading. 

\end{flushleft}

\subsection{Software}

You need to have access to a working version of Stata, (at least v
13.0).  Stata is installed on computers on Peabody campus, including
Wyatt 132, and on stations in the Peabody library. You are not
required to purchase Stata, but you will need to use it for class
assignments.

If you do purchase Stata, you will need Stata IC (standard
version). Vanderbilt has what's called a gradplan with Stata under
which you can purchase the software at greatly reduced prices.  Stata
SE is a more-powerful version of Stata that is useful for the larger
datasets many of you may be working with.


\subsection{Honor Code}
\label{sec:honor-code}

For this course, you are bound by the terms of the Peabody Honor
System. Any breach of academic honesty, including cheating,
plagiarism, or failing to report a known or suspected violation of the
Code will be reported to the Honor Council. In particular, papers must
assign credit to the sources you use. Material borrowed from
another--quotations, paraphrases, key words, or ideas--must be
credited following appropriate citation procedures (footnotes and
bibliography). As mentioned above, collaboration \textit{is} permitted
on assignments but \textit{is not} permitted on your summary paper and
codebook. 
 
If you have any doubts, please ask me for clarification. Uncertainty
about the application of the Honor Code does not excuse a violation.

\section{Schedule for Meetings}

The schedule for all class meetings is as follows:

 \subsection{January 8}

Topic: From conditional means to regression
 
\subsection{January 15}

Topic: Basics of regression in Stata

\subsection{January 22}

Topic: Reporting and explaining regression results

\subsection{January 29}

Topic: Using prediction to understand regression

\subsection{February 5}

Topic: binary and categorical predictors

\subsection{February 12}

Topic: Model design

\subsection{February 19}

Topic: Using simulation to understand regression

\subsection{February 26}

Topic: when good regressions go bad: diagnosing issues with OLS

\subsection{March 5}

\textit{No Class: Spring Break}

\subsection{March 12}

Topic: Advanced reporting

\subsection{March 19}

Topic: Panel data

\subsection{March 26}

Topic: Binary dependent variables

\subsection{April 2}

Topic: Creating replication files

\subsection{April 9}

Topic: advanced programming

  \end{document}