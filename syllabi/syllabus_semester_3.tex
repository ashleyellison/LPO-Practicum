\documentclass[10pt]{article}
\usepackage[margin= 1in]{geometry}
\usepackage{bibentry}
\usepackage{fourier}
\usepackage[colorlinks=true,
                      pdfstartview=FitV,
                      urlcolor=blue,
]{hyperref}
\usepackage{natbib}



\begin{document}

\bibliographystyle{apalike}
\nobibliography{may}

\thispagestyle{empty}%


\setlength{\parskip}{1ex plus 0.5ex minus 0.2ex}

\setcounter{secnumdepth}{-2}



\begin{flushleft}
Vanderbilt University\\Leadership, Policy and Organizations\\Class Number 9553\\ Maymester 2018\\
\end{flushleft}


\begin{center}
\Large{\textbf{Methods Practicum}}\\
\end{center}

\begin{flushleft}
William R. Doyle\\
Office: 207E Payne\\
Office Hours: We'll meet almost every day in May. If you need to see me at another time, let me know.\\
w.doyle@vanderbilt.edu\\
phone (615) 322-2904\\
\end{flushleft}

\section{Course Overview}%

The overview includes an introduction to the course, guidelines on grading, and required texts.

\subsection{Introduction}%
\begin{flushleft}

  This course is the final of a three semester series of courses
  designed to introduce you to the \textit{practice} of research,
  particularly the applied side of quantitative research. The goal of
  this course to help you to prepare a paper that can be presented at
  a major research conference and, hopefully, submitted to a journal
  for publication.


  During this session, we will work as a group to refine and improve
  your analysis. By the end of the term, each of you will have a
  completed paper suitable for presentation at a major research
  conference.

  Along the way, you will continue to develop the same set of skills
  you have been working on all year, along with some more specific
  set of techniques that I will spend the first week describing.


\end{flushleft}


\subsection{Grading}%
\begin{flushleft}
Evaluation for the course will be based on the following factors:

\begin{description}
\item[Final Paper] The final paper should be a publication-ready
  manuscript, reporting the results of the analysis conducted over the
  course of the year. You will need to turn in a replication file with
  this paper. This will be worth 100\% of your grade. The paper will
  be due at midnight on June 10. 


\item[Presentation] You will give a presentation based on your paper
  on May 24.  The presentation will be in preparation for the
  presentation you will give to all students and faculty in the
  fall. I will give you feedback on this presentation along with the
  rest of the class. 

\end{description}


\end{flushleft}
\subsection{Software}

Stata will again be the order of the day for this semester. 

\section{Schedule for Meetings}

Class will meet from 12-3 in Payne 108.  


\subsection{May 7}


Methods for Selection Bias: Propensity Score Matching

Readings: 

\begin{description}


\item \bibentry{agodini_are_2004}

\item \bibentry{ho_matching_2007}

\item \bibentry{imbens_nonparametric_2004}

\item \bibentry{moffitt_introduction_2004}

\item \bibentry{rubin_using_2001}


Introduction to Treatment Effects in Stata: Part 1
https://blog.stata.com/2015/07/07/introduction-to-treatment-effects-in-stata-part-1/

Introduction to Treatment Effects in Stata: Part 2
https://blog.stata.com/2015/08/24/introduction-to-treatment-effects-in-stata-part-2/



\end{description}

\subsection{May 8}

Limited Dependent Variables: Count variables and Truncated Outcomes

Introduction to Poisson Regression in Stata:

https://stats.idre.ucla.edu/stata/dae/poisson-regression/

Agresti, A. (2015). Foundations of linear and generalized linear models. John Wiley \& Sons, Chapter 7
(Available via web access from library)

Truncated Regression in Stata:

https://stats.idre.ucla.edu/stata/dae/truncated-regression/

Greene, Chapter 19

http://people.stern.nyu.edu/wgreene/Lugano2013/Greene-Chapter-19.pdf




\subsection{May 9}


Latent Class Analysis: Factor Analysis, PCA, etc

Readings:

\begin{description}

\item \bibentry{hagenars:2002aa}
\item \bibentry{ebrary-inc:2015aa}

\item \bibentry{bartholomew:2011aa}

\item \bibentry{mccutcheon:1987aa}

\end{description}


\subsection{May 10}

Instrumental Variables/ Regression Discontunity


\subsection{May 11}
\label{sec:may-13}

No Class: Commencement

\subsection{May 14}

Event History Analysis


\subsection{May 15}

Student Presentations: Amberly Dziezinski


\subsection{May 16}

Student Presentations: Karin Gegenheimer


\subsection{May 17}

Student Presentations: Richard Hall


\subsection{May 18}

Student Presentations: Shelby McNeill


\subsection{May 21}

Rapid-fire updates: Dziezinski and Gegenheimer


\subsection{May 22}

Rapid Fire updates: Hall and McNeill


\subsection{May 23} 
Code-A-Rama: We'll hold the class from 10-4, students can come in and
ask questions any time. 


\subsection{May 24}

Conference-Style Presentations



\end{document}


Missing data, multiple imputation

Readings:

\begin{description}

\item \bibentry{little_statistical_1987}

\item \bibentry{schafer_analysis_1997}

\item \bibentry{schafer_missing_2002}
  
\end{description}


Polychotomous Outcomes

Readings:


\begin{description}


\item \bibentry{long_regression_2005}

\item \bibentry{greene:2008aa}, Chapter 18

\item \bibentry{amemiya:1985aa}, Chapter 9

\end{description}

