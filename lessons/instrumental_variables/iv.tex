\documentclass[12 pt]{article}

\usepackage[pdftex]{graphicx}
\usepackage{amsmath}
\usepackage{bibentry}
\usepackage{ccaption}
\usepackage{fourier}
\usepackage{stata}
\usepackage{vmargin}%p.89 latex companion

\usepackage[colorlinks=true,
                      pdfstartview=FitV,
                      urlcolor=blue,
]{hyperref}


\newcommand{\boldbeta}{\boldsymbol{\beta}}
\newcommand{\boldy}{\boldsymbol{y}}
\newcommand{\boldX}{\boldsymbol{X}}
\newcommand{\boldx}{\boldsymbol{x}}
\newcommand{\boldz}{\boldsymbol{z}}
\newcommand{\boldgamma}{\boldsymbol{\gamma}}
\newcommand{\boldeps}{\boldsymbol{\epsilon}}

\usepackage{natbib}

\begin{document}

\thispagestyle{empty}%

\setpapersize{USletter}

\setmarginsrb{1in}{.5in}{.1in}{.5in}{0pt}{0mm}{0pt}{0mm}%

\setlength{\parskip}{1ex plus 0.5ex minus 0.2ex}

\setcounter{secnumdepth}{-2}



\begin{flushleft}
Vanderbilt University\\Leadership, Policy and Organizations\\Class Number 9553\\ Spring 2017
\end{flushleft}

\begin{center}
\textbf{Instrumental Variables}
\end{center}

Instrumental variables is predicated on the idea of a system of
equations. In our running example, we'll use a standard
earnings/education equation, where $x$ indicates the number of years
of education for an individual. We're interested in figuring out the
impact of an additional year of education on log earnings. 

The problem is the familiar one of selection bias, or
endogeneity. Individuals who are more likely to earn more may seek out
more years of education. This means that those individuals with more
education also have more earnings, but that it could be the case  
that giving more people more education would not increase earnings. 

To sort this out, we need to find something about the person's
environment or traits that could increase their years of education but
\textit{only} impacts earnings through the mechanism of education. 

If we had an experiment where some people were assigned to get more
education, then we could use the experimental assignment variable to
predict years of education, and there would be no reason that
experimental assignment predicted earnings, except through the
mechanism of education. 

\section{Two-Stage Least Squares}
\label{sec:two-stage-least}

The primary estimator for instrumental variables is Two Stage Least
Squares, often shattered to 2SLS. It works this way. Let's say we have
an endogenous regressor $x$ (years of education), an instrument $z$ and a
set of controls $\boldsymbol{c}$.

\begin{equation}
  \label{eq:1}
  x_i=\gamma_0+\gamma_1z_i+ \boldgamma \boldsymbol{c}_i + \mu_i
\end{equation}

This is called the first stage. We then take predictions of $x_i$ from
the first stage:

\begin{equation}
  \label{eq:1}
  \hat{x_i}=\gamma_0+\gamma_1z_i+ \boldgamma \boldsymbol{c}_i 
\end{equation}

And plug those into the equation we're actually interested in
estimating:

\begin{equation}
  \label{eq:2}
  y_i=\beta_0+\beta_1\hat{x}_i+\boldbeta \boldsymbol{c}_i+ \epsilon_i
\end{equation}

There's an adjustment that needs to be made to the standard errors in
the above regression, but with that, $\beta_1$ is our 2SLS estimator
of the impact of $x$ on $y$. Notice what the first stage does: it
takes everything out of $x$ that isn't a function of $z$ or $c$. This
``version'' of $x$ will have no correlation with the error term in the
second stage by construction, which is exactly what we want. 

IV estimates are easy to run-- the hard part is establish that the
instruments are valid. 

In a theoretical sense, what you need is an instrument that's truly
exogenous-- something that people are not choosing (or aren't choosing
with the outcome in mind), but affects assignment to treatment, and
ONLY affects the outcome through assignment to treatment. Most of the
time you need to know the conditions of the individuals in your sample
quite well to establish the validity of the instrument. 

In an empirical sense, there are two equally important properties of an
instrument. First, it must strongly predict assignment to
treatment. Second, it must not impact the outcome except through the
treatment variable. We can't always demonstrate both of these
assumptions are held.


\section{Establishing Endogeneity}
\label{sec:establ-endog}

Even suspected endogenity is sufficient to use IV, but you can try to
establish endogeniety by seeing whether the residuals from the first
stage predict the outcome in the basic form of the second stage. This
is known as the Durbin-Wu-Hausman test. If it's not significant (by a
lot), then there's less evidence of endogneity between $x$ and
$y$. However, this test is predicated on the assumption that the
instrument is valid.  

\section{Establishing the Strength of the Instrument(s)}
\label{sec:establ-strength-inst}

If the instruments aren't strongly related to the endogenous
regressor, we have ``weak'' instruments. These will result in both
high levels of variance in the second stage \textit{AND} biased
estimates in the second stage. That's bad. What you need to show is
that the instruments strongly predict the endogenous regressor. A
basic tests is the $F$ test for the first stage, testing the linear
restriction when the instruments are left out of this equation. A
more advanced test is the Stock \& Yogo minimum eigenvalue test. 

\section{Establishing Overidentification}
\label{sec:establ-over}

When you have just one instrument, the system of equations is said to
be ``exactly identified.'' When you have more instruments than
endogenous regressors, the system is ``overidientified.''
Overidentification can be helpful because it tests whether the
instruments have an impact on the outcome through another mechanism
beyond just the endogenous regressor. The basic Sargan test works by
taking the residuals from the first stage, then using them as a
predictor for the outcome. The $R^2$ from that regression times $N$ is
distributed $\chi^2$ with degrees of freedom equal to the number of
instruments minus 1.

 

\section{Local Average Treatment Effects}
\label{sec:local-aver-treatm}

Our understanding of IV has become more sophisticated over time. One
of the key insights from recent literature is that IV estimates are
Local Average Treatment Effects: they apply only to that part of the
population that was induced into treatment by the instrument. The
literature has identified the key assumptions that are crucial for this
understanding:

\begin{enumerate}
\item Independence assumption: $z$ is as good as randomly assigned,
  conditional on $x$
\item Exclusion: $y$ is impacted by $z$ only through $x$.
\item Monotonicity: $x$ can only stay the same or increase as a
  function of $z$, $z$ can never decrease $x$. (or vice-versa)
\end{enumerate}

What's important to remember in practice is that the external validity
of any IV estimate is limited to the group of people who could have
been induced by the instrument into treatment. 


\section{Angrist \& Pischke's Pragmatic Rules}
\label{sec:angrist--pischkes}

\begin{enumerate}
\item Report the first stage and think about whether it makes sense. 
\item REport the F Statistic on the excluded instruments. 
\item Pick your ingle best instrument and report just-identified
  estiumates using this one only. 
\item Check overidentified 2SLS estimates witrh LIML estimates. 
\item Look at the reduced form and make sure it makes sense,
  remembering that these are proportional to the causal effect of
  interest. 
\end{enumerate}

\end{document}

