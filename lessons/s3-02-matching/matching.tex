\documentclass[12pt]{article}
\usepackage[pdftex]{graphicx}
\usepackage{amsmath}
\usepackage{bibentry}
\usepackage{ccaption}
\usepackage{fourier}
\usepackage{stata}
\usepackage{vmargin}%p.89 latex companion

\usepackage[colorlinks=true,
                      pdfstartview=FitV,
                      urlcolor=blue,
]{hyperref}


\newcommand{\boldbeta}{\boldsymbol{\beta}}
\newcommand{\boldy}{\boldsymbol{y}}
\newcommand{\boldX}{\boldsymbol{X}}
\newcommand{\boldx}{\boldsymbol{x}}
\newcommand{\boldz}{\boldsymbol{z}}
\newcommand{\boldgamma}{\boldsymbol{\gamma}}
\newcommand{\boldeps}{\boldsymbol{\epsilon}}

\usepackage{natbib}

\begin{document}

\thispagestyle{empty}%

\setpapersize{USletter}

\setmarginsrb{1in}{.5in}{.1in}{.5in}{0pt}{0mm}{0pt}{0mm}%

\setlength{\parskip}{1ex plus 0.5ex minus 0.2ex}

\setcounter{secnumdepth}{-2}



\begin{flushleft}
Vanderbilt University\\Leadership, Policy and Organizations\\Class Number 9553\\ Spring 2017
\end{flushleft}

\begin{center}
\textbf{Methods for Matching}
\end{center}


Matching is in many ways a non-parametric form of
selection on observables. We're going to use all of the observed
information we have and make some assumptions about the distribution
of unobserved covariates. 

For this discussion, we're going to work with the idea of the
community college diversion effect. This is the impact of attending
community college on the probability of receiving a bachelor's
degree, among those students who said their goal is a bachelor's degree

Suppose that we divide groups into two different outcome groups.

Group $y_1$ denotes those students who were ``treated''-- those who
attended a community college

Group $y_0$ denotes those students who were not treated-- those who
attended a four year college. 

$z$ is a binary variable, where $z=1$ if a student is treated. 
$z=0$ otherwise. 

$y_i$ for any individual is:

\begin{equation*}
  y_i=
\begin{cases}
  y_{1i} \text{ if } z_i=1 \\
y_{0i} \text{ if } z_i=0
\end{cases}=
y_{0i}+(y_{1i}-y_{0i})z_i
\end{equation*}

The impact of participation is:
 
\begin{align*}
  E[y_i|z_i=1]-E[Y_i|z_i=0]&=E[y_{1i}|z_i=1]-E[y_{0i}|z_i=1]\\
                           &+E[y_{0i}|z_i=1]-E[y_{0i}|z_i=0]
\end{align*}

Under random assignment, selection bias goes away. This happens
because $z_i$ is independent of potential outcomes:

\begin{align*}
  E[y_i|z_i=1]-E[Y_i|z_i=0]&=E[y_{1i}|z_i=1]-E[y_{0i}|z_i=0]\\
                           &=E[y_{1I}|z_i=1]-E[y_{0i}|z_i=1]
\end{align*}

In experiments, the second part simplifies to: 

\begin{align*}
E[y_{0i}|z_i=1]-E[y_{0i}|z_i=1]&=E[y_{1i}-y_{0i}|z_i=1] \\
                               &=E[y_{1i}-y_{0i}]
\end{align*}

In regression and matching, $\boldx$ is  a series (or vector) of characteristics of students. 

We want to estimated treatment on the treated.

For each person, only $Y_1$ or $Y_0$ is observed, so we have a missing
data problem. 

\begin{equation*}
E(Y_1-Y_0|\boldx,z=1)=E(y_1|\boldx,z=1)-E(y_0|\boldx,z=1)
\end{equation*}

We don't observe $E(y_0|\boldx,z=1)$---this is the counterfactual
outcome. In an experimental study, we know that
$E(y_0|\boldx,z=1)=E(y_0|\boldx,z=0)$ because of randomization. In
observational studies, we need to make an assumption about the
information contained in $\boldx$ and its relationship with $y_0$. 

Information in $\boldx$ must be sufficient to meet the conditional independence assumption:

\begin{equation*}
(y_0, y_1\perp z)|\boldx
\end{equation*}

This is an \textit{extremely} strong assumption. What we're saying
here is that once we've got two students with the same distribution of
covariates, assignment to treatment or control is essentially
random. You have to think that there's no additional information that's
being supplied by whether the person gets the treatment. We can always
demonstrate that the distribution of covariates between the treatment
and control group looks similar after matching. This is a necessary
but NOT SUFFFICIENT condition to meet the conditional independence
assumption. 

We can then use the information in $x$ to create a propensity score
$p$, which denotes the probability of selection into the treatment or
control group. $p=pr(z=1|\boldx)$. This is estimated as the predicted value
from a model for binary data, typically a probit model. 

We then can estimate treatment on the treated by substitution:

\begin{equation*}
TT=E(y_1|\boldx,z=1)-E(y_0|\boldx,z=1)=E[y_1|p,z=1]-E[y_0|p,z=0]
\end{equation*} 

What does this mean in practice? We need a GREAT model for $p$, the
probability of selection. In fact, most of the original work on this
topics assumed $p$ was known, rather than estimated. Once we have an
estimate of $p$, we need to think hard
about how to match units based on this propensity score. 

There are multiple ways to create a matched control group to a
treatment group:

\begin{description}
\item[Exact Matching] Exact matching involves finding a unit in the control group that has
  \emph{exactly} the same values of the covariates as the treated
  unit. This works best in very large datasets with only a few
  covariates. Exact matching is typically the most rigorous approach,
  but can suffer from a lack of efficiency. As more variables are
  introduced, this approach also suffers from what's called the
  ``curse of dimensionality''. 

\item[Nearest Neighbor Matching] In nearest neighbor matching, the
  analyst finds an untreated unit that is most similar to the treated
  unit based on a measurement. In many cases, this measure is the
  propensity score, which is the predicted value of treatment for both
  the unmatched and matched unit, based on a regression for binary outcome with
  selection as the dependent variable. Nearest neighbor matching can
  use multiple controls per treatment. It can also use the same
  control twice for different treatment cases.  

\item[Regression conditional on the propensity score] Another
  technique is to estimate the propensity score, then include it as a
  control in the regression. Technically, this should get rid of
  selection bias in the parameter estimates. 

\item[Propensity Score Stratification] When stratifying based on the
  propensity score, the analyst estimates the propensity score as
  described above, then splits up the data based on levels of the
  propensity score. The analyst then runs the regression/does the
  comparison within each stratum, then averages the results, with
  weights (sometimes) based on the number of units in each stratum. 

\end{description}

We'll go over each of these methods and their advantages and
disadvantages. What you'll learn very quickly is that estimates vary
depending on the choice of estimator. The best practice for now is to
use a couple of different estimators as robustness checks. 

\section{Required Checks}
\label{sec:required-checks}

For the selection model, you need to demonstrate good model fit,
usually through the ROC curve stat, $c$. 

Any time you undertake matching, you must demonstrate that you have
achieved balance between the treatment and the control group. The
whole point of matching is that the treatment and control group are
very similar in terms of the \textit{distribution} of all observed covariates. This is an
empirical question that should be checked.

The standard comparison is to look at the mean and the variance of
covariates in the treatment and control group, both before and after
matching. Remember that the technical question you're asking is
whether the two groups are similar in terms of
distributions. Sometimes t-tests on the covariates between the
treatment and control group are used. These should be used quite
carefully, as a lack of significance in a t-test can come about
because of small sample sizes. One of the standard things that
matching algorithms do is to reduce sample size, so it's pretty easy
to get rid of statistical significance that way. Instead, use the
ratio of the variances, or the standardized difference. Whatever you
use, think about the problem of establishing similar distributions--
for continuous variables, kernel density plots can really help here. 


\section{Sensitivity  Analysis}

Propensity score matching has become widely used in education, and
most analysts are using it incorrectly. It  helps to reduce bias in
many cases, but it rarely meets the conditions that would be required
for it to be used for the purposes of causal inference. Once very
useful check on your results is to create bounds based on the
introduction of a theoretical, unobserved covariate with certain
characteristics. 

The idea here is to introduce a new variable, call it $c$. This
variable represents an unobserved covariate that can impact both
selection into treatment $pr(z|x,c)$ and the outcome $(y|x,c)$. By
creating a variable with known impacts on both selection and outcome,
we can come up with a sense of how ``powerful'' a missing covariate
would have to be in terms of both selecting into treatment and the
outcome. If it would require a very ``powerful'' covariate to change
the estimate, that's useful information. On the flip side, if a single
covariate with modest impacts on selection into treatment and the
outcome can cause results to vary, that should make us doubt the
results. 


\section{Final  Thoughts}
\label{sec:final-thoughts}

The promise of matching is that it can re-create experimental
estimates by creating ``as good as randomly assigned'' treatment and
control groups. The reality falls far short of this promise. It's much
better to think of matching as a means by which we can reduce bias in
our estimates due to a lack of balance and/or overlap in the
sample. In the end, there's never any such thing as a free lunch. 

\end{document}