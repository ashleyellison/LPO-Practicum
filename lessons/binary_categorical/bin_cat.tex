\usepackage[margin= 1in]{geometry}
\usepackage[pdftex]{graphicx}
\usepackage{amsmath}
\usepackage{bibentry}
\usepackage{ccaption}
\usepackage{fourier}
\usepackage{stata}
\usepackage[colorlinks=true,
                      pdfstartview=FitV,
                      urlcolor=blue,
]{hyperref}

\usepackage{natbib}

\begin{document}

\thispagestyle{empty}%


\setlength{\parskip}{1ex plus 0.5ex minus 0.2ex}

\setcounter{secnumdepth}{-2}



\begin{flushleft}
Vanderbilt University\\Leadership, Policy and Organizations\\Class Number 9952\\ Spring 2017
\end{flushleft}

\begin{center}
\textbf{Binary and Cateogrical Variables}
\end{center}


Binary and categorical variables can be a headache to work with. It's worth taking some time to think about each step 
with these kinds of variables in order to make sure that they are being reported effectively. 

\section{Coding}

First, it's worth thinking pretty carefully about how these variables
will be coded. Are you sure that they are mutually exclusive and
exhaustive? How about the numbers of categories? Are these appropriate
for the task at hand? Are they really categorical or can they be
thought of as ordered? How would you figure this out? 

In general, it's better to favor fewer categories, but you need to
make sure that your decisions reflect the important questions in your
theoeritcal framework. 

Below, I recode the race variables as they're constructed by NCES to
be more useful in our analysis. 

\begin{stlog}
  
\end{stlog}

\section{Binary Variables}

Binary variables must always be constructed to be directional. Never
have a binary variable for ``sex,'' always construct this kind of
binary variable as either ``male'' or ``female'' 



\section{Categorical Variables}

When running a model with categorical variables, Stata won't always
know what you're talking about. If the underlying variable is numeric,
it will simply include that variable as numeric. This is not
good. Instead, we need to use the \texttt{i.} formulation, which
specifies not only that a given variable is to be understood as a
factor variable, but also alows the user some fine-grained control
over how this will be constructed. 


\section{Reference Categories for Categorical Variables}


\end{document}