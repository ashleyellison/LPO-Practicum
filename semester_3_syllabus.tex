\documentclass[10pt]{article}
\usepackage[margin= 1in]{geometry}
\usepackage{bibentry}
\usepackage{fourier}
\usepackage[colorlinks=true,
                      pdfstartview=FitV,
                      urlcolor=blue,
]{hyperref}
\usepackage{natbib}



\begin{document}

\bibliographystyle{apalike}
\nobibliography{may}

\thispagestyle{empty}%


\setlength{\parskip}{1ex plus 0.5ex minus 0.2ex}

\setcounter{secnumdepth}{-2}



\begin{flushleft}
Vanderbilt University\\Leadership, Policy and Organizations\\Class Number 9553\\ Maymester 2017\\
\end{flushleft}


\begin{center}
\Large{\textbf{Methods Practicum}}\\
\end{center}

\begin{flushleft}
William R. Doyle\\
Office: 207E Payne\\
Office Hours: We'll meet almost every day in May. If you need to see me at another time, let me know.\\
w.doyle@vanderbilt.edu\\
phone (615) 322-2904\\
\end{flushleft}

\section{Course Overview}%

The overview includes an introduction to the course, guidelines on grading, and required texts.

\subsection{Introduction}%
\begin{flushleft}

  This course is the final of a three semester series of courses
  designed to introduce you to the \textit{practice} of research,
  particularly the applied side of quantitative research. The goal of
  this course to help you to prepare a paper that can be presented at
  a major research conference and, hopefully, submitted to a journal
  for publication.


  During this session, we will work as a group to refine and improve
  your analysis. By the end of the term, each of you will have a
  completed paper suitable for presentation at a major research
  conference.

  Along the way, you will continue to develop the same set of skills
  you have been working on all year, along with some more specific
  set of techniques that I will spend the first week describing.


\end{flushleft}



\subsection{Grading}%
\begin{flushleft}
Evaluation for the course will be based on the following factors:

\begin{description}
\item[Final Paper] The final paper should be a publication-ready
  manuscript, reporting the results of the analysis conducted over the
  course of the year. You will need to turn in a replication file with
  this paper. This will be worth 100\% of your grade. The paper will
  be due at midnight on June 10. 


\item[Presentation] You will give a presentation based on your paper
  on May 27.  The presentation will be in preparation for the
  presentation you will give to all students and faculty in the
  fall. I will give you feedback on this presentation along with the
  rest of the class. 

\end{description}


\end{flushleft}
\subsection{Software}

Stata will again be the order of the day for this semester. 

\section{Schedule for Meetings}

Class will meet from 12-3 in Payne 108.  



\subsection{May 8}

Missing data, multiple imputation

Readings:

\begin{description}

\item \bibentry{little_statistical_1987}

\item \bibentry{schafer_analysis_1997}

\item \bibentry{schafer_missing_2002}
  
\end{description}




\subsection{May 9}

Methods for Selection Bias: Propensity Score Matching

Readings: 

\begin{description}


\item \bibentry{agodini_are_2004}

\item \bibentry{ho_matching_2007}

\item \bibentry{imbens_nonparametric_2004}

\item \bibentry{moffitt_introduction_2004}

\item \bibentry{rubin_using_2001}

\end{description}


\subsection{May 10}


Latent Class Analysis: Factor Analysis, PCA, etc

Readings:

\begin{description}

\item \bibentry{hagenars:2002aa}
\item \bibentry{ebrary-inc:2015aa}

\item \bibentry{bartholomew:2011aa}

\item \bibentry{mccutcheon:1987aa}

\end{description}


\subsection{May 11}

Polychotomous Outcomes

Readings:


\begin{description}


\item \bibentry{long_regression_2005}

\item \bibentry{greene:2008aa}, Chapter 18

\item \bibentry{amemiya:1985aa}, Chapter 9

\end{description}

\subsection{May 12}
\label{sec:may-13}

No Class: Commencement

\subsection{May 15}

Event History Analysis



\subsection{May 16}
Student Presentations: Chris Bennett


\subsection{May 17}
Student Presentations: Michael Crouch


\subsection{May 18}
Student Presentations: Jennifer Darling-Aduana


\subsection{May 19}

Student Presentations: Walt Ecton

\subsection{May 22}
No Class


\subsection{May 23}
Student Presentations: Ashley Jones


\subsection{May 24} 
Catch-All: I will cover any topics that the class needs to on this
day. 


\subsection{May 25}

Code-A-Rama: We'll hold the class all day, students can come in and
ask questions any time. 

\subsection{May 26}

In-Class conference style presentations




\end{document}